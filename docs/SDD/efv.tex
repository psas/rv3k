\section{Earth Frame View (EFV) Module}
% TODO: Add Diagram

\subsection{EFV display}
  The Earth Frame display, takes in coordinates, map files, processed trajectory
  data, telemetry data, and recovery crew GPS data. It then outputs browser
  compatible code which renders a 3D map with overlay graphical
  representations of this data in an Earth Frame View.

\subsubsection{3D map rendering}
  The data must be in a form that the map rendering module can easily display.

  Map is rendered from local map files using a JavaScript library (TBD) which
  injects the map into an HTML div element.

\subsubsection{Trajectory path overlay}
  The pre-calculated trajectory is provided in a configuration file by the user.
  This trajectory object is pulled into the EFV module and rendered as an overlay
  on the map.

\subsubsection{Real-time vehicle trajectory location overlay}
  This module takes in the coordinates of the vehicle. The purpose for of the
  GPS data is to indicate the rockets current location on its trajectory.

  The current location of the vehicle is represented as a dot in the EFV display.

\subsubsection{Vehicle path overlay}
  As the vehicle moves on its trajectory path a red trail will be rendered to show
  the trajectory path the vehicle has traveled on.
  The logs the incoming coordinates of the vehicle. This is to insure
  that the rendering can display the path the vehicle has taken.

\subsubsection{Recovery Crew Location Indicator Overlay}
   The system takes in the GPS coordinates of the recovery crews. Earth frame
   view only needs to keep track of the current location of the recovery teams.

\subsubsection{Map angle views}

\subsection{Assets}

\subsubsection{Map files}
  Map files consist of collected terrain maps which are indexed by cartographic
  coordinates.

\subsection{Configuration files}
  Configurable JSON file that stores information about calculated trajectory,
  network ports, and location.

  %The system will display the trajectory of the of where the rocket has been.
